\documentclass{article}
\usepackage[utf8]{inputenc}
\textheight = 25cm 
\textwidth = 17cm
\topmargin = -3.0cm 
\oddsidemargin = 1.5cm
\usepackage{hyperref}
\hypersetup{
    colorlinks=true,
    linkcolor=blue,
    filecolor=blue,
    citecolor=black,      
    urlcolor=blue,
    }

\usepackage{float}
\usepackage{graphicx}

\usepackage{amsmath}
\usepackage{amssymb}
\usepackage{amsfonts}
\usepackage{mathtools, xparse}
\usepackage[shortlabels]{enumitem}

\usepackage[many]{tcolorbox}
\usepackage{lipsum}

\title{Tarea 2 Matemáticas Avanzadas de la Física}
\author{Cerritos Lira Carlos}
\date{25 de Marzo del 2020}

\begin{document}
\maketitle
\section*{Tercer parical}
\subsection*{Función Gamma}
Definición:
\[ \Gamma(p) = \int_0^\infty t^{p-1}e^{-t}dt, \quad p>0 \]
Propiedades:
\begin{align*}
    \Gamma(n+1) &= n! \\
    \Gamma(p+1) &= p\Gamma(p) \quad p>0 \\
    \Gamma(p) &= \frac{1}{p} \Gamma(p+1) \quad p <0 \\
    \Gamma(\tfrac{1}{2}) &= \sqrt{\pi} \\
    \Gamma(p)\Gamma(1-p) &= \frac{\pi}{sin(\pi p)}
\end{align*}
\subsection*{Función Beta}
Definición:
\[ B(p,q) = \int_{0}^1 x^{p-1}(1-x)^{q-1}dx \quad p,q > 0 \]
Propiedades:
\begin{align*}
    B(p,q) &= B(q,p) \\
    B(p,q) &= \frac{1}{a^{p+q-1}}\int_0^a y^{p-1}(a-y)^{q-1}dy \\
    B(p,q) &= 2\int_{0}^\frac{\pi}{2} (sin\theta)^{2p-1}(cos\theta)^{2q-1} d\theta \\
    B(p,q) &= \int_{0}^\infty \frac{y^{p-1}}{(1+y)^{p+q}}dy\\
    B(p,q) &= \frac{\Gamma(p)\Gamma(q)}{\Gamma(p+q)} 
\end{align*}
\begin{tcolorbox}
    \subsubsection*{Ejemplo: Péndulo:}
    Se tiene la relación:
    \begin{align*}
        \ddot{\theta} &= -\frac{g}{l}sin\theta \\
        T &= 2 \sqrt{\frac{l}{2g}}B(1/2,1/4)
    \end{align*}
\end{tcolorbox}

\subsection*{Función error}
Definición:
\[ erf(x) = \frac{2}{\pi} \int_{0}^x e^{-t^2}dt \]
Propiedades:
\begin{align*}
    erf(\infty) &= 1 \\
    \phi(x) &= \frac{1}{\sqrt{2\pi}} \int_{-\infty}^x e^{-\frac{t^2}{2}}dt \\
    etf(x) &= 2\phi(x\sqrt{2})-1 \\ 
    erfc(x) &= 1- erf(x) \\ 
    erf(x) &= \frac{2}{\sqrt{\pi}} (x-\frac{x^3}{3}+\frac{x^5}{5 \cdot 2!}-\frac{x^7}{7 \cdot 3!}+...) \quad |x|<< 1 \\
    erf(x) &= \frac{e^{-x^2}}{x\sqrt{\pi}} \left( 1-\frac{1}{2x^2} \right) \quad x>>1
  \end{align*}
Fórmula de Stirling.
\[ n! \sim n^n e^{-n} \sqrt{2\pi n} \quad n \to \infty \]
\[ \Gamma(p+1) \sim p^pe^{-p}\sqrt{2\pi p} \quad p \to \infty \]

\subsection*{Integrales y funciones elípticas}
\subsection*{Primera especie}
\subsubsection*{Forma de Legendre} 
\[ F(\phi,k) = \int_0^\phi \frac{d\theta}{\sqrt{1-k^2sin^2\theta}} \quad 0 \leq k \leq 1\]
se le llama a $\phi$ la amplitud y a $k$ el módulo.
\subsubsection*{Forma de Jacobi}
\[ F(\phi,k) =  \int_0^{sin\phi} \frac{dt}{\sqrt{1-t^2}\sqrt{1-k^2t^2}} \]
Definición:
\[ K(k) = F(\tfrac{\pi}{2},k) = \int_0^1 \frac{dt}{\sqrt{1-t^2}\sqrt{1-k^2t^2}} \quad 0 \leq k \leq 1\]
Propiedades:
\begin{align*}
    F(n\pi \pm \phi, k) &= 2nK(k) \pm F(\phi,k)b
\end{align*}

\subsection*{Segunda especia}
\subsubsection*{Forma de Legendre}
\[ E(\phi,k) = \int_{0}^\phi \sqrt{1-k^2sin^2\theta}d\theta \quad 0 \leq k \leq 1 \]
\subsection*{Forma de Jacobi}
\[ E(\phi,k) = \int_{0}^x \frac{\sqrt{1-k^2}t^2}{\sqrt{1-t^2}}dt \]
Definición forma completa:
\[ E(k) = E(\tfrac{\pi}{2}, k) \]

\subsection*{Expansión en series}
\[ K(k) = \frac{\pi}{2} \left(1 + \sum_{n=1}^\infty \left( \frac{(2n-1)!!}{(2n)!!} \right)k^{2n} \right) \]
\[ E(k) = \frac{\pi}{2} \left(1- \sum_{n=1}^\infty \left( \frac{(2n-1)!!}{(2n)!!} \right)^2 \frac{k^{2n}}{2n-1} \right) \]

\begin{tcolorbox}[breakable]
    \subsubsection*{Longitud de arco de una elipse}
    La elipse parametrizada por la función:
    \[ x(\theta) = asin\theta, \quad y(\theta) = bsin(\theta) \]
    se tiene la relación:
    \begin{align*}
        L 
        &= a\int_{0}^\phi \sqrt{1-\frac{a^2-b^2}{a^2}sin^2\theta}d\theta\\
        &= aE\left( \phi,\sqrt{\tfrac{a^2-b^2}{a^2}} \right)
    \end{align*}
    \subsubsection*{Péndulo}
    Encontrar el periodo para un péndulo cuando $\alpha	< \frac{\pi}{2}$
    
\end{tcolorbox}

\section*{Cuarto parcial}
\subsection*{Método de Frobenius}
Soluciones en términos de series de potencias:
\begin{align*}
    y'' + p(x)y'+q(x)y &= 0 \quad \text{$p,q$ analíticas en $x_0 \in R$}
\end{align*}
proponemos la solcuión:
\begin{align*}
    y(x) = \sum_{n=0}^\infty a_n(x-x_0)^n
\end{align*}
si $p,q$ no son analíticas en $x_0$ se puede tener:
\begin{enumerate}
    \item Punto singular regular si $(x-x_0)p(x), (x-x_0)^2q(x)$ analíticas en $x=x_0$
    \item Punto irregular caso contrario
\end{enumerate}
supondremos que $x_0=0$, nombraremos:
\[ f(x) = xp(x) \quad g(x)=x^2q(x) \]
tenemos entonces una nueva ecación diferencial:
\begin{align*}
    y''+\frac{f'(x)}{x}y'+\frac{g(x)}{x^2}y &= 0\\
    x^2y'' + xf(x)y'+g(x)y &= 0
\end{align*}
proponemos como solución:
\begin{align*}
    y(x) &= x^s\sum_{n=0}^\infty a_nx^n \quad a_0 \neq 0, s\in R 
\end{align*}
solucionaremos pirmero la ecuación:
\begin{align*}
    x^2y''+xf_0y'+g_0y &= 0
\end{align*}
la solción está dada por $y = x^s$, donde:
\begin{align*}
    s(s-1)+sf_0+g_0 &= 0
\end{align*}

\begin{tcolorbox}[breakable]
    \subsubsection*{Ejemplo}
    Resolveremos la ecuación diferencial:
    \[ xy'' + 2y'+xy = 0\]
\end{tcolorbox}

\subsection*{Teorema de Fuchs}
Dada la ecuación diferencial:
\[ x^2y''+xf(x)y'+g(x)y = 0 \]
$s_1 \geq s_2$ soluciones a la ecuación:
\[ s(s-1) + sf_0 + g_0 = 0 \]
siempre se tiene la solción:
\[ y_1(x) = x^{s_1}\sum_{n=0}^\infty a_nx^n \quad a_0 \neq 0 \]
se tienen las condicones:
\begin{enumerate}[a)]
    \item $s_1-s_2 \notin N_0 \implies y_2(x) = x^{s_2}\sum_{n=0}^\infty b_nx^n \quad b_0 \neq 0$
    \item $s_1-s_2 \in N \implies y_2(x) = x^{s_1+1}\sum_{n=0}^\infty b_nx^n + dy_1(x)log(x) \quad b_0 \neq 0, d \in R$
    \item $s_1=s_2 \implies y_2(x) = x^{s_1+1}\sum_{n=0}^\infty b_nx^n + y_1(x)log(x) $
\end{enumerate}    

\subsection*{Funciondes de Legendre}
\subsection*{Funciones de Sessel}
\subsection*{Funciones de Hermite, Lagre y Jacobi}

\end{document}

