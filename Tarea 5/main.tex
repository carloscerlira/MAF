\documentclass{article}
\usepackage[utf8]{inputenc}
\textheight = 25cm 
\textwidth = 18cm
\topmargin = -3.0cm 
\oddsidemargin = 0.5cm
\usepackage{hyperref}
\hypersetup{
    colorlinks=true,
    linkcolor=blue,
    filecolor=blue,
    citecolor=black,      
    urlcolor=blue,
    }

\usepackage{float}
\usepackage{graphicx}

\usepackage{amsmath}
\usepackage{amssymb}
\usepackage{amsfonts}
\usepackage{mathtools, xparse}
\usepackage[shortlabels]{enumitem}

\usepackage[many]{tcolorbox}
\usepackage{lipsum}

\title{Tarea 5 Matemáticas Avanzadas de la Física}
\author{Cerritos Lira Carlos}
\date{27 de Marzo del 2020}

\begin{document}
\maketitle
\subsection*{1.-}
En una ciudad se publican tres periódicos $B$ y $C$, y el $2\%$ lee los tres periódicos. Si se sabe que el total de habitanrtes en la ciuda , $A,B$ y $C$. Una encuesta reciente muestra que el 
$20\%$ de los habitantes adultos de la ciudad lee el periódico $A$, el $16\%$ lee el periódico $B$, 
el $14\%$ lee el periódicos $B$ y $C$, y el $2 \%$ lee los tres periódicos $A$ y $B$, el $5\%$ lee los 
periódicos $B$ y $C$, y el $2\%$ lee los tres periódicos. Si se sabe que el total de habitantes en la ciudad es de 
$20,000$ y se elige un adulto al azar,
\begin{enumerate}[a)]
    \item ¿Cuántos habitantes no leen ninguno de los periódicos?
    \item ¿Cuántos habitantes leen exactamente uno de los periódicos?
    \item Si $A$ y $B$ son periódicos que se publican por la mañana y $C$ se publica en la tarde, 
    ¿cuántos habitantes leen al menos un periódico de mañana y uno de tarde?
\end{enumerate}
\begin{tcolorbox}[breakable]
    \subsubsection*{a)}
    
    \subsubsection*{b)}

    \subsubsection*{c)}
\end{tcolorbox}

\subsection*{2.-}
Se consideran dos dados, $A$ y $B$. El dado $A$ tiene $4$ caras rojas y $2$ blancas, mientras que el dado 
$B$ tiene $2$ caras rojas y $4$ blancas. Se hace un volado de una mondeja justa. 
Si sale sol se usa el dado $A$, mientras que si sale águila se usa el dado $B$. Se repite sucesivamente el experimento.
\begin{enumerate}[a)]
    \item Demostrar que la probabilidad de que salga rojo en cualquier tirada es $\frac{1}{2}$
    \item Si en las dos primeras tiradas salen rojos, ¿cuál es la probabilidad de que en la ercera tirada salga un rojo también?
    \item Si en las dos primeras tiradas salen rojos, ¿cuál es la probabilidad de que el dado que esté usando las dos tiradas sea el dado $A$?
\end{enumerate}
\begin{tcolorbox}[breakable]
    \subsubsection*{a)}

    \subsubsection*{b)}

    \subsubsection*{c)}
\end{tcolorbox}

\subsection*{3.-}
Una urna contiene $5$ bolas blancas y $5$ bolas negras. Dos bolsas se sacan aleatoriamente (sin remplazo).
Si son iguales, ganamos $\$1.10$, pero si no son iguales perdemos $\$1.00$.
\begin{enumerate}[a)]
    \item Calcular la ganancia media esperada. ¿Es favorable el juego para el jugador?
    \item Calcular la varianza de la cantidad que se gana.
    \item Si llamamos $c$ a la cantidad que ganamos y $d$ a la cantidad que perdemos en el juego,
    ¿qué relación entre $c$ y $d$ debe ocurrir para que el juego sea favorable para el jugador?.
\end{enumerate}
\begin{tcolorbox}[breakable]
    \subsubsection*{a)}

    \subsubsection*{b)}

    \subsubsection*{c)}
\end{tcolorbox}

\subsection*{4.-}
El número de minutos $X$ que juega un jugador de básquetbol en un partido aleatorio es una variable aleatoria continua
con una función de densidad dada por:

\begin{enumerate}[a)]
    \item Comprobar que en efecto es una función de densidad y graficarla.
    \item ¿Cuál es la probabildiad de que el jugador juegue más de $15$ minutos?, ¿Y entre $20$ y $35$ minutis?,¿Y menos de $30$ minutos?,
    ¿Y más de $36$ minutos?
    \item Calcular $E(X)$ y $Var(X)$
\end{enumerate}
\begin{tcolorbox}[breakable]
    \subsubsection*{a)}

    \subsubsection*{b)}

    \subsubsection*{c)}
\end{tcolorbox}

\subsection*{5.-}
\begin{enumerate}[a)]    
    \item Sea $X$ una variable aleatorio normal de parámetros $\mu$ y $\sigma^2$. Demostrar que la variable aleatoria $Y = aX+b$ con $a>0,b\in R$, 
    es una variable aleatorio normal de parámetros $a\mu +b$ y $a^2\sigma^2$.
    \item Sea $X$ una variable aleatorio Poisson con parámetro $\lambda$. Calcular la probabilidad de que $X$ tome sólo valores pares, i.e. $P(X es par)$.
\end{enumerate}
\begin{tcolorbox}[breakable]
    \subsubsection*{a)}

    \subsubsection*{b)}
\end{tcolorbox}

\subsection*{6.-}
La cantidad de lluvia que cae en Ciudad de México anualmete es una variable aleatorio normal de media $840$ milímetros y desviación típica $150$ milímetros.
\begin{enumerate}[a)]
    \item ¿Cuál es la probabilidad de que en $2020$ llueva más de $900$ milímetros?
    \item ¿Cuál es la probabilidad de que en los próximos $7$ años haya exactamente $3$ años 
    donde se superen los $900$ milímetros?
\end{enumerate}
Expresar las probabilidades en términos de la función de distribución $\Phi(z)$ de la variable aleatoria 
normal estándar.
\begin{tcolorbox}[breakable]
    \subsubsection*{a)}

    \subsubsection*{b)}
\end{tcolorbox}

\subsection*{7.-}
Calcular el valor esperado $E(X)$ y la varianza $Var(X)$ de las siguientes variables aleatorias. Además, para cada una de ellas, mostrar una aplicación  
real en la cual se usen dichas variables aleatorias.

\begin{enumerate}[a)]
    \item $X$ es una variable aleatoria \textit{geométrica} con distribución de probabildiad:
    \[ P (X = n) = (1-p)^{n-1}p, \quad n = 1,2,...,0 \quad 0<p<1 \]
    \item $X$ es una variable aleatorio \textit{binominal negativa} con distribución de probabilidad:
    \[ P (X = n) =  \]
    \item $X$ es una variable aleatorio con \textit{distribución Gamma} con función de densidad:
    \[ f(x) = \frac{\lambda e^{-\lambda x}(\lambda x)^{\alpha-1}}{\Gamma(\alpha)}, \quad \alpha,\lambda > 0, \quad x>0 \]
    donde $\Gamma(\alpha)$ es la función \textit{Gamma}.
    \item $X$ es una variable aleatorio con \textit{distribución Cauchy} con una función de densidad:
    \[ f(x) = \frac{1}{\pi} \frac{1}{1+x^2}, \quad x \in R \]
    \item $X$ es una variable aleatorio con \textit{distribución Beta} con una función de densidad:
    \[ f(x) = \frac{1}{B(a,b)}x^{a-1}(1-x)^{b-1}, \quad a,b>0, \quad 0<x<1 \]
    donde $B(a,b)$ es la función Beta.
\end{enumerate}
\begin{tcolorbox}[breakable]
    \subsubsection*{a)}

    \subsubsection*{b)}
    
    \subsubsection*{c)}
    
    \subsubsection*{d)}
    
    \subsubsection*{e)}
\end{tcolorbox}



\end{document}
